\section{Marco Te\'orico}
\subsection{Deep Learning}
\paragraph{Funciones de Activaci\'on} Una funci\'on de activaci\'on tiene como prop\'osito introducir un elemento no lineal al output de un neuron. Esto es importante por que queremos que nuestros neuronas aprendan de una manera no lineal. Toda funci\'on de activaci\'on toma un solo numero y realiza una operaci\'on matem\'atica para darnos un output. Una de las funciones mas populares es la funci\'on de Sigmoid, es utilizada para tomar un numero y hacer un squash para que nos de un valor entre 0 y 1.

\paragraph{Las Neuronas de una Red}La estructura de una red neuronal es bien muy interesante. Tiene 3 capas que son, neuronas de entrada, neuronas escondidas y finalmente neuronas de salida. Las neuronas de entrada proporcionan informaci\'on sobre el mundo exterior a nuestra red. Tambi\'en contiene un node conocido como el bias node que por default tiene un valor de 1. Nuestras neuronas escondidas tienen como prop\'osito realizar c\'alculos y transfieren informaci\'on desde los nodos de entrada a los nodos de salida. Los nodos de salida, finalmente, realizan pocos c\'alculos y mas que nada muestran el resultado de nuestra red al mundo exterior.

\paragraph{Gradient Descent, Backpropagation y feedforwarding} Empecemos hablando sobre que es el prop\'osito de gradient descent en nuestra red. El punto de el algoritmo de gradient descent es minimizar el error de nuestra red. Como vamos a hacer esto? Pues para nuestra sigmoid function, es tan simple como calcular su derivada. Backpropagation utilizara nuestra gradiente, multiplicada por el error del neuron y seteara el delta peso de nuestra conexi\'on. Finalmente, feedforwarding se refiere a la manera en la que una red propaga y calcula su output. Feedforward es cuando el flujo de informaci\'on se mueve en un solo sentido. Osea que el error se calcula inmediatamente en cada neuron y no necesitan propagar la informaci\'on a otros neuronas cada una de las neuronas calculan de parte de lo que les dice la funci\'on de sigmoid.
